%\todo[inline]{Conclude, like normal}
A quantum computer is the ultimate goal of researchers within \gls{cqc2t}, and devices and experiments are continuously being devised to this end. Devices have been composed of an \gls{set} for spin-dependent readout, a must for a spin-based quantum computer, and experiments are constructed at cryogenic temperatures, and spin qubit manipulation is performed with series of microwave pulses. Before you perform operations on a qubit, it needs to start in a well defined state, and current methods rely on pre-selection of data, based on the criterion of having an electron present at the commencement of the operations. This is an inefficient use of time, and this can be improved by incorporating a digital feedback loop that will perform a strong measurement on the electron spin, and signal the continuation of an experiment.

In my literary review, I have covered some of the fundamental concepts governing the devices being used currently in this research, as well as some other research into the use of digital feedback in quantum systems.
The experimental design explored how the devices were being used together, and where the solution of this thesis will be place in the larger system. It also details some proposed solutions to the lack of digital feedback.
Finally, the prospective plan lists various tasks that are to be completed, from a high-level perspective by the completion of Thesis B.