%\todo[inline]{Conclude, like normal}
A quantum computer is the ultimate goal of researchers within \gls{cqc2t}, and devices and experiments are continuously being devised to this end. Devices have been composed of an \gls{set} for spin-dependent readout, a necessity for a spin-based quantum computer, and experiments are conducted at cryogenic temperatures. Further, spin-qubit manipulation is performed with series of \gls{nmr} and \gls{esr} pulses. Before qubit operations can be performed, the system needs to start in a well defined state, and current methods rely on post-selection of data, based on the criterion of having an electron present on the donor at the commencement of the experiment. This is an inefficient use of time, which can be improved upon by incorporating a digital feedback loop that will perform a quantum steering on the electron spin, and trigger the start of an experiment.

In this thesis, we have presented a method for applying electron spin steering through a weakly projective measurement. After a pre-defined amount of time, the measurement becomes projective enough such that the electron is initialised in the spin-down state with high probability. Noteworthy results include a total experiment fidelity of 99.8\%, which was within the limits of accuracy of the measurement. A large contrast was presented for the fidelity as a function of steering time, showing a strong dependence that matched well with theory.

%In my literary review, I have covered some of the fundamental concepts governing the devices being used currently in this research, as well as some other research into the use of digital feedback in quantum systems.
%The experimental design explored how the devices were being used together, and where the solution of this thesis will be place in the larger system. It also details some proposed solutions to the lack of digital feedback.