Quantum computers present a new paradigm for solving various tasks. A general purpose quantum computer has been proved to exceed a classical computer in certain applications, such as sorting a database, and is inherently better at modelling quantum effects and performing simulations, which has large implications for the drug industry and chemical engineering broadly. This report aims to introduce the concept of a quantum computer, and current attempts at creating and controlling qubits, quantum bits of information. The body of work in this thesis is geared towards improving qubit state initialization fidelity, which would make all further experiments more efficient and cost effective. The prime way of improving initialization is to incorporate a feedback loop that will self-correct for any potential error before proceeding with any further operations on the qubit. Digital systems have been used to great effect in closing the loop, and this is the approach taken through this report.