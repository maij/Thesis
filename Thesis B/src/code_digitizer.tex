%% Process buffer; checking for threshold
switch exp_phase
    case 'init'
        inEmpty = any(buff_data(samplesPerRecord+1:2*samplesPerRecord) < emptyLevel);
        if (inEmpty)
           emptied = 1; 
        elseif (~start_wait && (emptied || buffersCompleted > 10))
            % Once we have been to empty, and now we're back again start counting buffers
           start_wait = 1;
        end
        
        is_peak = any(buff_data(1:samplesPerRecord) > readout_threshold);
        % This will increment buff_count when false, and reduce to 0 when true
        buff_count = buff_count + (~is_peak & start_wait) - is_peak*buff_count;

        % Found enough buffers with no current blip
        if (buff_count >= buff_target)
            exp_phase = 'readout';
        end
        % Waited long enough, just plunge
        if buffersCompleted >= max_init_buffs
            fprintf('Warning: Reached maximum number of buffers during initialization\n');
            exp_phase = 'readout';
        end
        % If we're about to change to plunge, then trigger the PB
        if strcmp(exp_phase,'readout')
            buff_count = 0;
            if calllib('spinapi64','pb_start')<0; disp(['PB Error: ',calllib('spinapi64','pb_get_error')]); end
            plunged = true; 
        end
    case 'readout'
         % Look for rising/falling edges on channel C to detect when to perform readout
        
         channelA = buff_data(1:samplesPerRecord);
         channelC = buff_data(samplesPerRecord+1:samplesPerRecord*2);
         
         readoutPhase = channelC < trigLevel;
         isReadoutPhase = any(readoutPhase);
         
		 % Convert output_data to a cell array for each of the n nuclear reads
         if isReadoutPhase
            % Append current buffer to output data for this readout
            output_data{readoutsCompleted+1} = [output_data{readoutsCompleted + 1} channelA(readoutPhase)];
         end
         
         if isReadoutPhase && plunged
            % Falling edge of channel C, coming out of plunge
            plunged = false;
         elseif ~plunged && any(~readoutPhase)
            % Rising edge of channel C, going into plunge
            readoutsCompleted = readoutsCompleted+1;
            plunged = true;
         end
         
         % Once we reach required number of electron spin readouts, finish up OR if we reach a set maximum number of buffers
         buff_count = buff_count + 1;
         if readoutsCompleted >= totRecords || buff_count >= max_readout_buffs
             captureDone = true;
         end
    otherwise
        fprintf('Error: Undefined experiment phase : %s', exp_phase)
        captureDone = true;
end %switch